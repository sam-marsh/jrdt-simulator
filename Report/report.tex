\documentclass[a4paper]{article}

\usepackage{amsthm}
\usepackage{amsmath}
\usepackage{amssymb}
\usepackage{multirow}
\usepackage{graphicx}
\usepackage[margin=2.5cm]{geometry}
\usepackage{hyperref}
\usepackage{color}
\usepackage{titlesec}
 \usepackage{url}
 
\titlespacing\subsubsection{0pt}{2pt plus 4pt minus 2pt}{2pt plus 2pt minus 2pt}

\newcommand{\code}{\url}

\title{\vspace{-5ex}Assignment 01}
\author{{Reliable Data Transport \hspace{1cm} Candidate: 153728}}
\date{}

\begin{document}
\maketitle
\vspace{-4ex}

\section{Stop-and-Wait}

The stop-and-wait protocol is implemented using a finite-state machine, described in the \code{Sender} class by an enum \code{SenderState} containing the values \code{WAIT_MSG} and \code{WAIT_ACK}. In the \code{WAIT_MSG} state, the sender waits for a message from the application layer. In the \code{WAIT_ACK} state, a packet is currently `in-transit'. That is, the sender is waiting for acknowledgement from the receiver. An integer field \code{seq} keeps track of the current packet. Since the stop-and-wait protocol is an alternating bit protocol, \code{seq} only holds either of the values $0$ or $1$ (this is done by setting its value modulo $2$). In the \code{WAIT_MSG} state this field is used for the sequence number of the next message from the application layer. 


\section{Go-Back-N}

Upon receiving a new message from the application layer, if there is free space in the buffer, a new packet is created and stored with the message data and a sequence number of \code{nextSeqNum}. Then if the sequence number is within the window range (\code{base} $\leq$ \code{nextSeqNum} $<$ \code{base} + \code{WINDOW_SIZE}), the newly created packet is sent to the client and if the packet is the first in the window, the timer is started. In both cases, \code{nextSeqNum} is then incremented.

Upon receiving a new packet from the other host, the packet is first checked to be corrupt - if it is, it is ignored and no further action is taken. Otherwise the acknowledgement number of the packet is extracted - all packets with a sequence number below or equal to this have been successfully received by the receiver. In the initial implementation, this means the base could be immediately shifted up to one more than the received acknowledgement number. However this implementation involves buffering, so the actions taken at this point had to be modified:

The Stop-and-Wait protocol is a specific case of the Go-Back-N protocol described below. Indeed, the Stop-and-Wait protocol can be recovered from setting the \code{WINDOW_SIZE} to $1$.


\end{document}

