\documentclass[a4paper]{article}

\usepackage{amsthm}
\usepackage{amsmath}
\usepackage{amssymb}
\usepackage{multirow}
\usepackage{graphicx}
\usepackage[margin=2.5cm]{geometry}
\usepackage{hyperref}
\usepackage{color}
\usepackage{titlesec}
 \usepackage{url}
 
\titlespacing\subsubsection{0pt}{2pt plus 4pt minus 2pt}{2pt plus 2pt minus 2pt}

\newcommand{\code}{\url}

\title{\vspace{-5ex}Assignment 01}
\author{{Reliable Data Transport \hspace{1cm} Candidate: 153728}}
\date{}

\begin{document}
\maketitle
\vspace{-4ex}

\section{Protocols}

\subsection{Stop-and-Wait}

\subsubsection{Sender}

The stop-and-wait protocol is implemented using a finite-state machine, described in the \code{Sender} class by an enum \code{SenderState} containing \code{WAIT_MSG} and \code{WAIT_ACK}. In the \code{WAIT_MSG} state, the sender waits for a message from the application layer. In the \code{WAIT_ACK} state, a packet is currently `in-transit'. That is, the sender is waiting for acknowledgement from the receiver.


\subsection{Go-Back-N}

\section{Source code}

\section{Design}

\end{document}

